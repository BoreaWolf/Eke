% Riccardo Orizio
% 7 January 2016
% Description of the quality factor of every solution

\documentclass[10pt, letterpaper]{article}

\usepackage[utf8]{inputenc}
\usepackage{mathtools}
\usepackage{amsmath}
\usepackage{amssymb}
\usepackage{color,soul}
\usepackage[usenames,dvipsnames]{xcolor}
\usepackage{graphicx}
\graphicspath{ {../images/} }
\usepackage{float}

% Every list item will start with a dash instead of a dot
% \def\labelitemi{--}

% Personal commands
% Highlighting text with different colors
\newcommand{\hlc}[2]{\sethlcolor{#1} \hl{#2}}
\newcommand{\news}[1]{\hlc{green}{#1}}
\newcommand{\todelete}[1]{\hlc{red}{#1}}
\newcommand{\extras}[1]{\newline\fbox{\parbox{\textwidth}{\hlc{orange}{#1}}}\newline}
\newcommand{\todo}[1]{\hlc{Emerald}{TODO:#1}}

\title{Quality Factors and Incentives}
\author{Riccardo Orizio}
\date{7 January 2016}

\begin{document}

\maketitle

\section{Abstract}
Description of Quality Factors, Solution and Time Window, and how they are used
in the incentives computation process.

\section{Statistics}
From every solution we retrieve these statistics:
\begin{itemize}
	\item \emph{Cost} of the solution, calculated as sum of the travelling time
		of every driver
	\item Number of \emph{Drivers} used
	\item Stats on every Time Window
		\begin{itemize}
			\item Time spent inside the time window, divided by \emph{Serving
				Time} and \emph{Driving Time}\footnote{This value is an
				underestimation of the real driving time passed inside the time
				window itself}
			\item $\emph{Time ratio}\ =\ \frac{Active\ Time}{TW\ Span\ Time}$
			\item Number of \emph{Clients} served
			\item Number of \emph{Drivers} passed by
		\end{itemize}
	\item Stats on every Driver
		\begin{itemize}
			\item Time stats:
				\begin{itemize}
					\item \emph{Serving Time}
					\item \emph{Driving Time}
					\item \emph{Active Time} = Serving + Driving Time
					\item \emph{Remaining Time} = Working $-$ Active Time
					\item $\emph{Time ratio}\ =\ \frac{Active\ Time}{Working\ Time}$
				\end{itemize}
			\item Capacity stats:
				\begin{itemize}
					\item \emph{Occupied} capacity
					\item \emph{Remaining} capacity
					\item $\emph{Capacity Ratio} = \frac{Occupied}{Available}$
				\end{itemize}
		\end{itemize}
\end{itemize}

\section{Quality Factors}
For computing the \emph{Incentives} of our time windows, we are going to make
use of two types of \emph{Quality Factors}:
\begin{description}
	\item[Solution Quality Factor] \hfill \\
		This Quality Factor represent the goodness of a Solution, calculated
		taking in account different values from the solution itself and how good
		it is exploited
	\item[Time Window Quality Factor] \hfill \\
		This describes how good a Time Window is, in respect of the solution
		based on it and the probability of the starting problem
\end{description}

\subsection{Solution Quality Factor}
From the previous statistics we can extract a Solution Quality Factor, based on
them with exception for the time windows stats.

These last ones statistics are not really interesting if we want to get a
solution quality factor because we cannot really change them or in any way try
to influence how the drivers pass in them, they are a fixed input data and these
statistics are useful only to see how they are being used, nothing more.
Moreover the TWs that we are using right now are completly random, so it's
reasonable to see some TWs partially of completly not used: with smart and well
balanced time windows we can expect an improvement in quality on these values.

The Solution Quality Factor is calculated as follows:
\begin{equation}
	\begin{aligned}
		Q_{S}\ = &\	\alpha \cdot Time\ Cost \cdot Solution\ Cost\ + \\
				 &\	\beta \cdot Driver\ Cost \cdot Drivers\ + \\
			 	 &\	\gamma \cdot Driver\ Usage\ Cost \cdot Drivers\ Stats
	\end{aligned}
\end{equation}
having
\begin{itemize}
	\item $Solution\ Cost$: cost of the solution
	\item $Time\ Cost$: cost per minute
	\item $Drivers$: number of drivers used
	\item $Driver\ Cost$: cost per driver
	\item $Drivers\ Stats$: a value indicating how the Drivers have been used,
		underling how much of their time and capacity are wasted.

		This value is computed as described here:
		\[Drivers\ Stats\ = \delta\ Capacity\ Ratio + \xi\ Time\ Ratio\]
		with
		\\
		\[Capacity\ Ratio = 
			\frac
				{\sum_{i\ in\ Used\ Drivers} Remaining\ Capacity_{i}}
				{Number\ of\ Drivers}
		\]
		\\
		\[Time\ Ratio =
			\frac
				{\sum_{i\ in\ Used\ Drivers} Remaining\ Time_{i}}
				{Number\ of\ Drivers}
		\]
	\item $Driver\ Usage\ Cost$: cost of drivers while not working
\end{itemize}
%	Note that every parameter showed in the previous equations
%	(such as
%	$\alpha\ \beta\ \gamma\ \delta\ \xi\ $
%	Time Cost, Driver Cost and Driver Usage Cost)
%	have to be tested!

The lower the Quality Factor value is, the better the solution will be.
If we compare two solutions, the best one is the one with the \emph{lower}
Quality Factor value: this because the Quality Factor is focused on the costs of
the solution, as showed before.

\subsection{Time Window Quality Factor}
After calculating the Solution Quality Factors, we are combining them and then
retrieving the Time Window Quality Factors, one per time windows involved in the
process.

This Quality Factor is calculated differently from the previous one and it is
nothing else than the weighted average of the Solution Quality Factors (of the
current time window) having as weight the probability associated to the
problem\footnote{There is no need to divide by \(\sum_{p\ in\ P_{t}} prob_{p}\)
because (by construction) that sum is going to be exactly one.}
\[Q_{T} = \sum_{s\ in\ S_{t}} prob_{p} \cdot Q_{s} \]
$p$ is the problem associated to the solution $s$.

\section{Incentives}
In this section we discuss how we extract the \emph{Incentive} values starting
from the Time Window Quality Factors.

First of all we order the time windows by their Quality Factor\footnote{Remember
that the lower the value of Q is, the best it is.} and then we divide them in
different sets, depending on their quality, in particular:
\begin{description}
	\item[Best] \hfill \\
		Best time window for the current customer
	\item[Good] \hfill \\ 
		We are going to save the time windows with a Quality Factor between 0\%
		and $\eta\%$ from the best solution found
	\item[Nothing] \hfill \\ 
		Time windows that do not require much attention: these time windows
		don't penalize too much the company but they don't even give profit to
		it
	\item[Penalty] \hfill \\ 
		Bad solutions for the company, so if the customer still wants to choose
		to be served from these time windows, we are going to give him a sort of
		tax for picking one of the worst solutions. The Quality Factor of these
		time windows are over a certain threshold $\vartheta$
\end{description}

Depending on some possibilities, we give our incentives in different ways;
instead the penalties are always treated in the same way.

\subsection{Penalties}
We know that the time windows in this set have their quality factor worse than
$\vartheta\%$ in comparison to the best solution.

The penalty that we assign to these time windows is directly proportional to the
difference between its quality factor and the average quality factor of every
time window.

\[Penalty_{t} = BUDGET\ \cdot 
				5 \cdot 
				\frac{Quality_{t}\ -\ Avg_{TW}}{Avg_{TW}}
\]


\subsection{Incentives}
Since the budget has to be splitted between both good and best time windows we
need a way to divide it properly.

First of all we calculate the average of the good and best\footnote{It's very
unlikely to find more than one best time window, so calculating also the mean of
the best time windows is not going to influence the final result.} time windows
and than we split the budget weighing two times the good part, so the
best part is going to receive more budget.

\[Best\ Budget\ = BUDGET\ \cdot 
				  \left( 1 - \frac
								{ Avg_{Best} }
								{ Avg_{Best} + 2 \cdot Avg_{Good} }
				  \right)
\]

\[Good\ Budget\ = BUDGET\ \cdot 
				  \left( 1 - \frac
								{ 2 \cdot Avg_{Good} }
								{ Avg_{Best} + 2 \cdot Avg_{Good} }
				  \right)
\]

Note: the `$1 - \ldots$' part is needed because we are working with Quality
Factors that are better as long as their value is lower than others.

For both sets we are going to subdivide again the budget between the time
windows forming each set and, as already done with the penalties, we assign the
incentives with a direct proportionality law:

\[Good\ Incentive_{t}\ = Good\ Budget \cdot
						 \left(
							 1 - \frac{Quality_{t}}{Sum_{Good}}
						 \right)
\]

\[Best\ Incentive_{t}\ = Best\ Budget \cdot \frac{Quality_{t}}{Sum_{Best}}
\]

Note that if there is only one good time window, the previous law will assign to
it an incentive equal to $0.0$: in this particular case we are going to
assign the whole good budget to that time window. \\

All of this just described is true only if we have at least one good time
window, otherwise we are just going to give an incentive to the best time window
directly from the whole budget, as described by this equation:

\[Best\ Incentive_{t}\ = \frac{Budget}{2} \cdot \frac{Quality_{t}}{Sum_{Best}}
\]

In this case we decided to not assign all the budget but only a part of it
because, since we only have one time window that will receive an incentive, it
seems too much giving all the budget to it and, according also to \emph{Campbell
and Savelsbergh, 2005}, incentives don't need to have high values to pay off.

\section{Parameters}
After some tests and reasoning, we decided to set the parameters to the
following values:
\begin{description}
	\item[\emph{$\alpha$}] = 0.5 \hfill \\
		High emphasis on the solution cost
	\item[\emph{$\beta$}] = 0.1 \hfill \\
		Not caring too much about how many drivers are required, but
	\item[\emph{$\gamma$}] = $1-\alpha-\beta$ = 0.4 \hfill \\
		Caring how bad the drivers required are used
	\item[\emph{$\delta$}] = 0.5
	\item[\emph{$\xi$}] = $1-\delta$ = 0.5 \hfill \\
		Same weight on how the drivers are used, divided between capacity and
		time not used ratio
	\item[\emph{Cost Driver}] = 50.0 \hfill \\
		Renting cost of a new driver
	\item[\emph{Cost Driver Usage}] = 10.0 \hfill \\
		Cost of time not used for each driver
	\item[\emph{Cost Time}] = 20.0 \hfill \\
		Cost of time when drivers are working
	\item[\emph{$\vartheta$}] = 0.05 \hfill \\
		Penalty threshold over which we are going to assign a penalty
	\item[\emph{$\eta$}] = 0.02 \hfill \\
		Good threshold: this time window deserve an incentive because they are
		almost as good as the best one
	
\end{description}

\end{document}
